% \section{Question 1: X-Ray Computed Tomography – Radon Transform}

% A $128 \times 128$ Shepp–Logan phantom was generated and treated as the image $f(x,y)$. 
% The coordinate origin was logically placed at the center pixel.

% \subsection*{(a) Implementation of \texttt{myXrayIntegration()}}

% The function \texttt{myXrayIntegration(f, t, theta\_deg, delta\_s)} computes the line integral 
% of image intensities along a line parameterized by $t$ and $\theta$.

% Implementation details:

% \begin{itemize}
% \item The angle \texttt{theta\_deg} is converted to radians.
% \item A sampling variable \texttt{s} is created from $-N$ to $N$ with step size \texttt{delta\_s}.
% \item For each value of \texttt{s}, corresponding $(x,y)$ coordinates are computed.
% \item \texttt{map\_coordinates()} is used to interpolate image values.
% \item The interpolated values are summed and multiplied by \texttt{delta\_s}.
% \end{itemize}

% \paragraph{Choice of $\Delta s$:}

% Different values were tested: $0.1, 0.5, 1, 3, 10$ pixel units.

% Observations:

% \begin{itemize}
% \item Very small $\Delta s$ increases computation with little improvement.
% \item $\Delta s = 0.5$ or $1$ gives stable and smooth results.
% \item Large $\Delta s$ (3 or 10) causes rough and blocky sinograms.
% \end{itemize}

% \paragraph{Interpolation Scheme:}

% Bilinear interpolation (\texttt{order=1}) was used because:

% \begin{itemize}
% \item It provides smooth transitions between pixels.
% \item Nearest-neighbor interpolation produces jagged artifacts.
% \item Higher-order interpolation increases computation unnecessarily.
% \end{itemize}

% \subsection*{(b) Implementation of \texttt{myXrayCTRadonTransform()}}

% The function \texttt{myXrayCTRadonTransform()} computes the Radon transform 
% over discrete values:

% \begin{itemize}
% \item \texttt{t = -90 to 90} with step size $\Delta t = 5$
% \item \texttt{theta = 0 to 175} with step size $\Delta \theta = 5$
% \end{itemize}

% For every $(t, \theta)$ pair, the function calls \texttt{myXrayIntegration()} 
% and stores the result in a 2D array.

% \subsection*{(c) Comparison of Different $\Delta s$ Values}

% Radon transforms were generated for:

% \begin{itemize}
% \item $\Delta s = 0.5$
% \item $\Delta s = 1$
% \item $\Delta s = 3$
% \end{itemize}

% Additionally, 1D plots of the sinogram values were examined for:

% \begin{itemize}
% \item $\theta = 0^\circ$
% \item $\theta = 90^\circ$
% \end{itemize}

% Observations:

% \begin{itemize}
% \item Smaller $\Delta s$ produces smoother 1D curves.
% \item Larger $\Delta s$ produces visible roughness due to coarse sampling.
% \item Among the tested values, $\Delta s = 0.5$ appears smoothest.
% \end{itemize}

% A smoothness score was computed using the function \texttt{apply\_prior()}, 
% which evaluates local intensity differences. Lower values indicate smoother sinograms.

% Results show:

% \begin{itemize}
% \item $\Delta s = 0.5$ gives the lowest smoothness score.
% \item $\Delta s = 3$ and $10$ significantly increase roughness.
% \end{itemize}

% \subsection*{(d) Choice of $\Delta t$ and $\Delta \theta$ in Scanner Design}

% \paragraph{$\Delta \theta$:}

% \begin{itemize}
% \item Smaller $\Delta \theta$ improves angular resolution.
% \item Too small increases radiation exposure and scan time.
% \item A moderate small value is preferred.
% \end{itemize}

% \paragraph{$\Delta t$:}

% \begin{itemize}
% \item Smaller $\Delta t$ improves spatial sampling.
% \item Very small $\Delta t$ increases noise sensitivity.
% \item Larger $\Delta t$ introduces discretization artifacts.
% \end{itemize}

% Thus, both parameters must balance resolution, noise, and acquisition cost.

% \subsection*{(e) Design Considerations for ART Reconstruction}

% \paragraph{Number of Pixels and Pixel Size:}

% \begin{itemize}
% \item More pixels improve spatial resolution.
% \item However, computational cost increases significantly.
% \item Smaller pixels receive fewer photons, increasing noise variance.
% \end{itemize}

% A moderate grid resolution is therefore preferred.

% \paragraph{Effect of $\Delta s$:}

% \begin{itemize}
% \item $\Delta s \ll 1$ pixel width:
%     \begin{itemize}
%     \item More accurate integration.
%     \item Higher computational cost.
%     \item Minimal improvement beyond a certain point.
%     \end{itemize}
% \item $\Delta s \gg 1$ pixel width:
%     \begin{itemize}
%     \item Underestimates line integrals.
%     \item Produces blocky reconstruction.
%     \item Slows ART convergence.
%     \end{itemize}
% \end{itemize}

% \subsection*{Summary}

% The experiments demonstrate that:

% \begin{itemize}
% \item $\Delta s$ should be comparable to pixel width.
% \item $\Delta t$ and $\Delta \theta$ should be small but not excessively small.
% \item There is a trade-off between resolution, noise sensitivity, radiation dose, and computational complexity.
% \end{itemize}





\section{Question 1: X-Ray Computed Tomography – Radon Transform}

A $128 \times 128$ Shepp–Logan phantom was generated and treated as the image $f(x,y)$. 
The coordinate origin was logically placed at the center pixel.

\subsection*{(a) Implementation of \texttt{myXrayIntegration()}}

The function \texttt{myXrayIntegration(f, t, theta\_deg, delta\_s)} computes the line integral 
of image intensities along a line parameterized by $t$ and $\theta$.

Implementation details:

\begin{itemize}
\item The angle \texttt{theta\_deg} is converted to radians.
\item A sampling variable \texttt{s} is created from $-N$ to $N$ with step size \texttt{delta\_s}.
\item For each value of \texttt{s}, corresponding $(x,y)$ coordinates are computed.
\item \texttt{map\_coordinates()} is used to interpolate image values.
\item The interpolated values are summed and multiplied by \texttt{delta\_s}.
\end{itemize}


    % Log prior to calc smoothness... 0.1
    % 0.007664408181411028
    % Log prior to calc smoothness... 0.5
    % 0.007663613111978756
    % Log prior to calc smoothness... 1
    % 0.007665728259402043
    % Log prior to calc smoothness... 3
    % 0.007924217046504343
    % Log prior to calc smoothness... 10
    % 0.013104456194064591
% smoothness score to bne kept upto 8 digits after decimal only

\paragraph{Interpolation Scheme:}

Bilinear interpolation (\texttt{order=1}) was used because:

\begin{itemize}
\item It provides smooth transitions between pixels.
\item Nearest-neighbor interpolation produces jagged artifacts.
\item Higher-order interpolation increases computation unnecessarily.
\end{itemize}

\subsection*{(b) Implementation of \texttt{myXrayCTRadonTransform()}}

The function \texttt{myXrayCTRadonTransform()} computes the Radon transform 
over discrete values:

\begin{itemize}
\item \texttt{t = -90 to 90} with step size $\Delta t = 5$
\item \texttt{theta = 0 to 175} with step size $\Delta \theta = 5$
\end{itemize}

For every $(t, \theta)$ pair, the function calls \texttt{myXrayIntegration()} 
and stores the result in a 2D array.

\subsection*{(c) Comparison of Different $\Delta s$ Values}

\paragraph{Choice of $\Delta s$:}

Different values were tested: $0.1, 0.5, 1, 3, 10$ pixel units.

Observations:

\begin{itemize}
\item Very small $\Delta s$ increases computation with little improvement.
\item Infact $\Delta s = 0.1$ performed worse than $\Delta s = 0.5$.
\item This is due to overfitting by interpolation.
\item $\Delta s = 0.5$ or $1$ gives stable and smooth results.
\item Large $\Delta s$ (3 or 10) causes rough and blocky sinograms.
\end{itemize}
\begin{table}[H]
\centering
\begin{tabular}{|c|c|c|}
\hline
$\Delta s$ & Smoothness Score \\
\hline
0.1 & 0.00766441 \\
0.5 & 0.00766361 \\
1 & 0.00766573 \\
3 & 0.00792422 \\
10 & 0.01310446 \\
\hline
\end{tabular}
\caption{Smoothness scores for different $\Delta s$ values.}
\end{table}


\begin{figure}[H]
\centering
\begin{tabular}{c c}

$\Delta s = 0.5$ &
\includegraphics[width=0.5\textwidth]{./../q1/output/radon_transform_delta_s_0.5.png} \\[0.6cm]

$\Delta s = 1$ &
\includegraphics[width=0.5\textwidth]{./../q1/output/radon_transform_delta_s_1.png} \\[0.6cm]

$\Delta s = 3$ &
\includegraphics[width=0.5\textwidth]{./../q1/output/radon_transform_delta_s_3.png} \\

\end{tabular}
\end{figure}



Additionally, 1D plots of the sinogram values were examined for:

\subsubsection*{1D Projection Comparisons for Different $\Delta s$}

\begin{figure}[H]
\centering
\begin{tabular}{c c c}

$\Delta s$ & $\theta = 0^\circ$ & $\theta = 90^\circ$ \\[0.4cm]

0.5 &
\includegraphics[width=0.32\textwidth]{./../q1/output_part_c/projection_theta_0_delta_s_0.5.png} &
\includegraphics[width=0.32\textwidth]{./../q1/output_part_c/projection_theta_90_delta_s_0.5.png} \\[0.6cm]

1 &
\includegraphics[width=0.32\textwidth]{./../q1/output_part_c/projection_theta_0_delta_s_1.png} &
\includegraphics[width=0.32\textwidth]{./../q1/output_part_c/projection_theta_90_delta_s_1.png} \\[0.6cm]

3 &
\includegraphics[width=0.32\textwidth]{./../q1/output_part_c/projection_theta_0_delta_s_3.png} &
\includegraphics[width=0.32\textwidth]{./../q1/output_part_c/projection_theta_90_delta_s_3.png} \\

\end{tabular}
\end{figure}


Observations:

\begin{itemize}
\item Smaller $\Delta s$ produces smoother 1D curves.
\item Larger $\Delta s$ produces visible roughness due to coarse sampling.
\item Among the tested values, $\Delta s = 0.5$ appears smoothest.
\end{itemize}

A smoothness score was computed using the function \texttt{apply\_prior()}, 
which evaluates local intensity differences. Lower values indicate smoother sinograms.

Results show:

\begin{itemize}
\item $\Delta s = 0.5$ gives the lowest smoothness score.
\item $\Delta s = 1$ and $3$ give slightly higher scores, indicating more roughness.
\end{itemize}


\subsection*{(d) Choice of $\Delta t$ and $\Delta \theta$ in Scanner Design}

\paragraph{$\Delta \theta$:}

\begin{itemize}
\item Smaller $\Delta \theta$ improves angular resolution.
\item Too small increases radiation exposure and scan time.
\item A moderate small value is preferred.
\end{itemize}

\begin{figure}[H]
\centering
\begin{tabular}{c c}

$\Delta \theta = 0.1$ &
\includegraphics[width=0.45\textwidth]{./../q1/output_delta_theta/radon_transform_delta_theta_0.1.png} \\[0.6cm]

$\Delta \theta = 0.5$ &
\includegraphics[width=0.45\textwidth]{./../q1/output_delta_theta/radon_transform_delta_theta_0.5.png} \\[0.6cm]

$\Delta \theta = 2.5$ &
\includegraphics[width=0.45\textwidth]{./../q1/output_delta_theta/radon_transform_delta_theta_2.5.png} \\

\end{tabular}
\end{figure}


\paragraph{$\Delta t$:}



\begin{itemize}
\item Smaller $\Delta t$ improves spatial sampling.
\item Very small $\Delta t$ increases noise sensitivity.
\item Larger $\Delta t$ introduces discretization artifacts.
\end{itemize}

\begin{figure}[H]
\centering
\begin{tabular}{c c}

$\Delta t = 0.1$ &
\includegraphics[width=0.45\textwidth]{./../q1/output_delta_t/radon_transform_delta_t_0.1.png} \\[0.6cm]

$\Delta t = 0.5$ &
\includegraphics[width=0.45\textwidth]{./../q1/output_delta_t/radon_transform_delta_t_0.5.png} \\[0.6cm]

$\Delta t = 2.5$ &
\includegraphics[width=0.45\textwidth]{./../q1/output_delta_t/radon_transform_delta_t_2.5.png} \\

\end{tabular}
\end{figure}



Thus, both parameters must balance resolution, noise, and acquisition cost.

\subsection*{(e) Design Considerations for ART Reconstruction}

\paragraph{Number of Pixels and Pixel Size:}

\begin{itemize}
\item More pixels improve spatial resolution.
\item However, computational cost increases significantly.
\item Smaller pixels receive fewer photons, increasing noise variance.
\end{itemize}

A moderate grid resolution is therefore preferred.

\paragraph{Effect of $\Delta s$:}

\begin{itemize}
\item $\Delta s \ll 1$ pixel width:
    \begin{itemize}
    \item More accurate integration.
    \item Higher computational cost.
    \item Less energy per pixel so noise has greater effect.
    \end{itemize}
\item $\Delta s \gg 1$ pixel width:
    \begin{itemize}
    \item Underestimates line integrals.
    \item Produces blocky reconstruction.
    \item Slows ART convergence.
    \end{itemize}
\end{itemize}

\subsection*{Summary}

The experiments demonstrate that:

\begin{itemize}
\item $\Delta s$ should be comparable to pixel width.
\item $\Delta t$ and $\Delta \theta$ should be small but not excessively small.
\item There is a trade-off between resolution, noise sensitivity, radiation dose, and computational complexity.
\end{itemize}
