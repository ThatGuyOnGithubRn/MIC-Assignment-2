\section{Question 2: X-Ray Computed Tomography — Reconstruction by Filtered Backprojection (FBP)}

\subsection*{(a) Filter Implementation and Reconstruction Results}

Three filters were implemented in the Fourier domain using FFT/IFFT:

\begin{itemize}
\item Ram-Lak
\item Shepp-Logan
\item Cosine
\end{itemize}

Each filter was evaluated at two cutoff frequencies:
\[
L = \omega_{\max}, \qquad L = \frac{\omega_{\max}}{2},
\]
where $\omega_{\max}$ is the highest discrete frequency determined by the sampling of the sinogram.

The unfiltered backprojection was also computed for comparison.

\begin{figure}[H]
\centering

\begin{tabular}{c}
\includegraphics[width=0.35\textwidth]{../q2/output_filtered/recon_no_filter.png} \\
\textbf{No Filtering}
\end{tabular}

\vspace{0.5cm}
% --- Row 2: Ram-Lak ---
\begin{tabular}{cc}
\includegraphics[width=0.45\textwidth]{../q2/output_filtered/recon_ram_lak_L_0.500.png} &
\includegraphics[width=0.45\textwidth]{../q2/output_filtered/recon_ram_lak_L_0.250.png} \\
Ram-Lak ($L = \omega_{\max}$) & Ram-Lak ($L = \omega_{\max}/2$)
\end{tabular}

\vspace{0.5cm}
\end{figure}

\begin{figure}[p]
\centering

% --- Row 3: Shepp-Logan ---
\begin{tabular}{cc}
\includegraphics[width=0.45\textwidth]{../q2/output_filtered/recon_shepp_logan_L_0.500.png} &
\includegraphics[width=0.45\textwidth]{../q2/output_filtered/recon_shepp_logan_L_0.250.png} \\
Shepp-Logan ($L = \omega_{\max}$) & Shepp-Logan ($L = \omega_{\max}/2$)
\end{tabular}

\vspace{0.5cm}

% --- Row 4: Cosine ---
\begin{tabular}{cc}
\includegraphics[width=0.45\textwidth]{../q2/output_filtered/recon_cosine_L_0.500.png} &
\includegraphics[width=0.45\textwidth]{../q2/output_filtered/recon_cosine_L_0.250.png} \\
Cosine ($L = \omega_{\max}$) & Cosine ($L = \omega_{\max}/2$)
\end{tabular}

\caption{Reconstruction results using different FBP filters and cutoff frequencies.}
\end{figure}




\paragraph{Observations and Discussion}

\begin{itemize}

\item \textbf{Unfiltered Backprojection:}  
The unfiltered reconstruction appears overly bright and severely blurred. Edges are poorly defined and fine structures are lost. This behavior is expected because simple backprojection introduces a $1/|\omega|$ frequency weighting, which excessively emphasizes low frequencies while failing to properly compensate high-frequency attenuation. As a result, the image lacks sharpness and contrast.

\item \textbf{Ram-Lak Filter:}  
The Ram-Lak filter corresponds to the ideal ramp $|\omega|$ and restores the correct frequency weighting required by the inversion formula.  

For $L=\omega_{\max}$, edges are clearly delineated and the reconstruction is the sharpest among all cases. However, noticeable streak artifacts appear, particularly outside the circular support, due to high-frequency amplification combined with finite angular sampling.  

When $L=\omega_{\max}/2$, high frequencies are truncated. The reconstruction becomes smoother and streak artifacts are reduced, but edge definition is slightly degraded.

\item \textbf{Shepp-Logan Filter:}  
The Shepp-Logan filter multiplies the ramp by a sinc term, which gradually attenuates higher frequencies.  

Compared to Ram-Lak at full bandwidth, the reconstruction shows reduced streaking while preserving most structural details. For the reduced cutoff, the image becomes smoother, with slightly diminished edge contrast but fewer high-frequency artifacts.

\item \textbf{Cosine Filter:}  
The cosine filter applies a cosine apodization to the ramp, leading to stronger high-frequency suppression.  

For $L=\omega_{\max}$, the reconstruction is smoother than both Ram-Lak and Shepp-Logan, though mild ringing patterns are visible. When $L=\omega_{\max}/2$, the image becomes noticeably smoother, and edges appear softer due to the stronger attenuation of high frequencies.

\end{itemize}

\paragraph{Influence of the Cutoff Frequency}

Across all filters, increasing $L$ improves edge sharpness by retaining more high-frequency components. However, it also enhances streak artifacts and oscillatory patterns caused by discretization and limited angular sampling. Reducing $L$ suppresses these artifacts at the cost of spatial resolution.

Overall, the Ram-Lak filter with full bandwidth yields the sharpest reconstruction but exhibits the strongest artifacts. The Shepp-Logan filter provides a balanced compromise between resolution and stability, while the cosine filter produces the smoothest images with reduced edge contrast.



\subsection*{(b) Effect of Gaussian Blurring on Reconstruction}
\subsection*{(c) RRMSE as a Function of Cutoff Frequency}